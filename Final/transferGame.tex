\documentclass{article}
\usepackage{amsmath, amssymb, mathtools, verbatim, nicefrac}
\usepackage[margin=0.5in]{geometry}
\setlength{\parindent}{0pt}
\renewcommand{\t}[1]{\text{#1}}
\newcommand{\N}{\mathbb{N}}
\newcommand{\sumx}[2]{\sum\limits_{#1}^{#2}}
\newcommand{\ord}{\text{ord}}
\newcommand{\frp}{\text{FRP}}
\newcommand{\R}{\Rightarrow\,}
\newcommand{\B}[1]{\textbf{#1}}
\newcommand{\I}{\bigcap}
\newcommand{\U}{\bigcup}
\newcommand{\V}{\vert}
\newcommand{\F}[1]{\lfloor#1\rfloor}
\newcommand{\C}[1]{\lceil#1\rceil}
\renewcommand{\t}[1]{\text{#1}}
\newcommand{\lra}[1]{\langle#1\rangle}
\newcommand{\len}[1]{\vert#1\vert}
\newcommand{\num}{\text{num}}
\newcommand{\game}[3]{\begin{array}{@{}*6r}(#1, & #2, & #3)\end{array}}
\newcommand{\gcong}{\cong\:}
\newcommand{\nfrc}[2]{\nicefrac{#1}{#2}}
\begin{document}
\begin{center}
  21--499 \\
  Kartik Sabharwal (ksabharw)
\end{center}
\paragraph{Three Point Situation, 1-Position Proof, Case 8}\mbox{}\\
We want to prove the following proposition:
\begin{equation*}
  P(n) : \game{0}{6n + 8}{0} \gcong *1 \enspace \text{where} \enspace n \geq 0
\end{equation*}
We will use strong induction to prove the above statement.

\bigskip
\textbf{Base Cases:} \\
Using our program, we have verified that $P(n)$ holds for
$n \in \{0, 1, 2, 3\}$. \\
The fact that $\game{0}{2}{0} \gcong *1$ will also be useful.

\bigskip
\textbf{Induction Hypothesis:} \\
Assume for some $k \geq 1$ that 
$\forall \: n < k, \: P(n) \: \text{holds}$.

\bigskip
\textbf{Induction Step:} \\
We need to prove that $P(k)$ holds. In other words, we need to show:
\begin{equation*}
  \game{0}{6k + 8}{0} \gcong *1
\end{equation*}

\bigskip
\textit{Simplifying the problem,}\\
To show that $\game{0}{6k+8}{0}$ has nimber $1$, it suffices to show
the following:
\begin{enumerate}
  \item In one move, we can reach a state with nimber $0$
  \item In one move, we cannot reach a state with nimber $1$
\end{enumerate}

\bigskip
\textit{Looking at the initial state's children,}\\
Notice that in one move from the original state we can reach any state
of the form:
\begin{equation*}
  \game{m}{6k+8-m}{0} \text{where} \: 1 \leq m \leq 3k + 4 
  \end{equation*}
the remaining states that we can reach from the original state are of
the form:
\begin{equation*}
  \game{0}{6k+8-m}{m} \text{where} \: 1 \leq m \leq 3k + 4
\end{equation*}
Since the latter form is symmetric to the former, focusing our analysis
on the former is enough to complete the proof.

\bigskip
\textit{When} $m = 3k + 4$:\\
Observe that when $m = 3k + 4$, we obtain the state
$\game{3k+4}{3k+4}{0}$ which clearly has nimber $0$,
so we have proved the first fact we need. \\

\bigskip
\textit{When} $m$ \textit{is odd and} $1 \leq m \leq 2k + 2$:\\
When $m$ is an odd number satisfying $1 \leq m \leq 2k + 2$,
notice that
\begin{equation*}
  \game{m+1}{3k+4-\frac{m+1}{2}}{3k+4-\frac{m+1}{2}} \gcong *0
\end{equation*}
Work one step backwards from this to get
\begin{equation*}
  \game{m}{3k+5-\frac{m+1}{2}}{3k+4-\frac{m+1}{2}} \gcong *1
\end{equation*}
Observe that
\begin{equation*}
  \game{m}{6k+8-m}{0} \xrightarrow{\text{one move}}
  \game{m}{3k+5-\frac{m+1}{2}}{3k+4-\frac{m+1}{2}}
\end{equation*}
It follows that the minimum excluded value of the nimbers of
all the child states of $\game{m}{6k+8-m}{0}$ cannot be $1$, and
thus the nimber of this state cannot be 1.

\newpage
\textit{When} $m$ \textit{is even and} $1 \leq m \leq 2k + 2$:\\
When $m$ is an even number satisfying $1 \leq m \leq 2k + 2$,
observe that 
\begin{equation*}
  \game{m}{6k+8-m}{0} \xrightarrow{\text{one move}}
  \game{m}{6k+8-2m}{m}
\end{equation*}
We can replace $m$ with $2i$ where $1 \leq i \leq k + 1$ to get
\begin{equation*}
  \game{2i}{6k+8-2i}{0} \xrightarrow{\text{one move}}
  \game{2i}{6k+8-4i}{2i}
\end{equation*}
In the resultant state, the difference between the left and middle
values, also the difference between the right and middle values, is
\begin{equation*}
  6(k - i) + 8
\end{equation*}
Which intuitively suggests that
\begin{equation*}
  \game{2i}{6k+8-4i}{2i} \gcong \game{0}{6(k - i) + 8}{0} \:
  \text{for $1 \leq i \leq k + 1$}
\end{equation*}
and although we lack a rigorous argument for this we will assume it is
true. \\
Our knowledge that $\game{0}{2}{0} \gcong *1$ and also our induction
hypothesis, tell us:
\begin{equation*}
  \game{0}{6(k - i) + 8}{0} \gcong *1 \:
  \text{for $1 \leq i \leq k + 1$}
\end{equation*}
Consequently, $\game{2i}{6k+8-4i}{2i} \gcong *1$. \\
It follows that the minimum excluded value of the nimbers of
all the child states of $\game{m}{6k+8-m}{0}$ cannot be $1$, and
thus the nimber of this state cannot be 1.

\bigskip
\textit{When} $m \geq 2k + 3$:\\
If $m \geq 2k + 3$, it is clear that
\begin{equation*}
  \game{m}{m}{6k+8-2m} \gcong *0
\end{equation*}
Working backwards from this, we get that
\begin{equation*}
  \game{m}{m+1}{6k+7-2m} \gcong *1
\end{equation*}
We know that
\begin{equation*}
  \game{m}{6k+8-m}{0} \xrightarrow{\text{one move}}
  \game{m}{m+1}{6k+7-2m}
\end{equation*}
It follows that the minimum excluded value of the nimbers of
all the child states of $\game{m}{6k+8-m}{0}$ cannot be $1$, and
thus the nimber of this state cannot be 1.

\bigskip
\textit{Conclusion}:\\
$1$ must be the mex of the nimbers of the children of $\game{0}{6k+8}{0}$,
so we have proved that $\game{0}{6k+8}{0} \gcong *1$.
\newpage

\paragraph{Three Point Situation, 1-Position Proof, Case 6}\mbox{}\\
Assume the same inductive `skeleton' and reasoning style as the previous case.
This proof is mainly an illustration of the differences between the two cases:

\bigskip
Consider the initial position
\begin{equation*}
  \game{0}{6n+6}{0} \: \text{where} \: n \geq 0
\end{equation*}

\medskip
The first move will take the initial position to a position of the form
\begin{equation*}
  \game{m}{6n+6-m}{0} \: \text{where} \: 1 \leq m \leq 3n+3
\end{equation*}

\bigskip
\underline{If $m = 3n$:} \\
This is the state $\game{3n}{3n}{0}$ which clearly has nimber $0$.

\bigskip
\underline{If $m \leq 2n + 2$ and $m$ is odd:} \\
Then, consider the following move sequence,
\begin{align*}
  & \game{m}{6n+6-m}{0} \\
  \xrightarrow{\text{one possible child}} \quad & 
  \game{m}{3n+4-\frac{m+1}{2}}{3n+3-\frac{m+1}{2}} \\
  \xrightarrow{\text{only child}} \quad & 
  \game{m+1}{3n+3-\frac{m+1}{2}}{3n+3-\frac{m+1}{2}} \\
\end{align*}

\bigskip
\underline{If $m \leq 2n + 2$ and $m$ is even:} \\
Consider the following move sequence:
\begin{align*}
  & \game{m}{6n+6-m}{0} \\
  \xrightarrow{\text{one possible child}} \quad & 
  \game{m}{6n+6-2m}{m} \\
  \cong \quad & 
  \game{0}{6n+6-3m}{0} \\
\end{align*}
Since $m$ is even, we can write it as $2i$ for some $1\leq i\leq n+1$.
\begin{align*}
  & \game{0}{6n+6-3m}{0} \\
  \cong \quad & \game{0}{6n+6-6i}{0} \\
  \cong \quad & \game{0}{6(n-i)+6}{0} \\
\end{align*}
By our induction hypothesis, $\game{0}{6(n-i)+6}{0}$ has nimber $1$.

\bigskip
\underline{If $m \geq 2n + 3$:} \\
The move sequence below works:
\begin{align*}
  & \game{m}{6n+6-m}{0} \\
  \xrightarrow{\text{one possible child}} \quad & 
  \game{m}{m+1}{6n+5-2m} \\
  \xrightarrow{\text{only child}} \quad & 
  \game{m}{m}{6n+6-2m} \\
\end{align*}

\bigskip
Which provides us with everything we need to fill in the proof
skeleton.
\newpage

\paragraph{Three Point Situation, 1-Position Proof, Case 10}\mbox{}\\
Assume the same inductive `skeleton' and reasoning style as the previous case.
This proof is mainly an illustration of the differences between the two cases:

\bigskip
Consider the initial position
\begin{equation*}
  \game{0}{6n+10}{0} \: \text{where} \: n \geq 0
\end{equation*}

\medskip
The first move will take the initial position to a position of the form
\begin{equation*}
  \game{m}{6n+6-m}{0} \: \text{where} \: 1 \leq m \leq 3n+3
\end{equation*}

\bigskip
\underline{If $m = 3n$:} \\
This is the state $\game{3n}{3n}{0}$ which clearly has nimber $0$.

\bigskip
\underline{If $m \leq 2n + 3$ and $m$ is odd:} \\
Then, consider the following move sequence,
\begin{align*}
  & \game{m}{6n+10-m}{0} \\
  \xrightarrow{\text{one possible child}} \quad & 
  \game{m}{3n+6-\frac{m+1}{2}}{3n+5-\frac{m+1}{2}} \\
  \xrightarrow{\text{only child}} \quad & 
  \game{m+1}{3n+5-\frac{m+1}{2}}{3n+5-\frac{m+1}{2}} \\
\end{align*}

\bigskip
\underline{If $m \leq 2n + 3$ and $m$ is even:} \\
Consider the following move sequence:
\begin{align*}
  & \game{m}{6n+10-m}{0} \\
  \xrightarrow{\text{one possible child}} \quad & 
  \game{m}{6n+10-2m}{m} \\
  \cong \quad & 
  \game{0}{6n+10-3m}{0} \\
\end{align*}
Since $m$ is even, we can write it as $2i$ for some $1\leq i\leq n+1$.
\begin{align*}
  & \game{0}{6n+10-3m}{0} \\
  \cong \quad & \game{0}{6n+10-6i}{0} \\
  \cong \quad & \game{0}{6(n-i)+10}{0} \\
\end{align*}
By our induction hypothesis, $\game{0}{6(n-i)+10}{0}$ has nimber $1$.

\bigskip
\underline{If $m \geq 2n + 4$:} \\
The move sequence below works:
\begin{align*}
  & \game{m}{6n+10-m}{0} \\
  \xrightarrow{\text{one possible child}} \quad & 
  \game{m}{m+1}{6n+9-2m} \\
  \xrightarrow{\text{only child}} \quad & 
  \game{m}{m}{6n+10-2m} \\
\end{align*}

\bigskip
Which provides us with everything we need to fill in the proof
skeleton.
\newpage

\paragraph{Three Point Situation, 1-Position Proof, Case 7}\mbox{}\\
Consider the initial position
\begin{equation*}
  \game{0}{6n+6}{0} \: \text{where} \: n \geq 0
\end{equation*}

\medskip
The first move brings $\game{0}{6n+7}{0}$ to a state of the form
$\game{m}{6n+7-m}{0}$, where $1 \leq m \leq 3n+3$. \\
If $m = 3n + 3$,
\begin{align*}
  & \game{3n+3}{3n+4}{0} \\
  \xrightarrow{\text{only child}} \quad & 
  \game{3n+3}{3n+3}{1}
\end{align*}

\bigskip
\underline{If $m \leq 2n + 2$ and $m$ is odd:} \\
Then, consider the following move sequence,
\begin{align*}
  & \game{m}{6n+7-m}{0} \\
  \xrightarrow{\text{one possible child}} \quad & 
  \game{m}{\frac{6n+7-m}{2}}{\frac{6n+7-m}{2}}
\end{align*}

\bigskip
\underline{If $m \leq 2n + 2$ and $m$ is even:} \\
Consider the following move sequence:
\begin{align*}
  & \game{m}{6n+7-m}{0} \\
  \xrightarrow{\text{one possible child}} \quad & 
  \game{m}{6n+7-2m}{m} \\
  \cong \quad & 
  \game{0}{6n+7-3m}{0} \\
\end{align*}
Since $m$ is odd, we can write it as $2i$ for some $1\leq i\leq n+1$.
\begin{align*}
  & \game{0}{6n+7-3m}{0} \\
  \cong \quad & \game{0}{6n+7-6i}{0} \\
  \cong \quad & \game{0}{6(n-i)+7}{0} \\
\end{align*}
By our induction hypothesis, $\game{0}{6(n-i)+6}{0}$ has nimber $0$.

\bigskip
\underline{If $m \geq 2n + 3$:} \\
The move sequence below works:
\begin{align*}
  & \game{m}{6n+7-m}{0} \\
  \xrightarrow{\text{one possible child}} \quad & 
  \game{m}{m}{6n+7-2m}
\end{align*}

\bigskip
Which provides us with everything we need to fill in the proof
skeleton.

\paragraph{Three Point Situation, 1-Position Proof, Case 9}\mbox{}\\
Consider the initial position
\begin{equation*}
  \game{0}{6n+6}{0} \: \text{where} \: n \geq 0
\end{equation*}

\medskip
The first move brings $\game{0}{6n+7}{0}$ to a state of the form
$\game{m}{6n+7-m}{0}$, where $1 \leq m \leq 3n+3$. \\
If $m = 3n + 3$,
\begin{align*}
  & \game{3n+3}{3n+4}{0} \\
  \xrightarrow{\text{only child}} \quad & 
  \game{3n+3}{3n+3}{1}
\end{align*}

\bigskip
\underline{If $m \leq 2n + 2$ and $m$ is odd:} \\
Then, consider the following move sequence,
\begin{align*}
  & \game{m}{6n+7-m}{0} \\
  \xrightarrow{\text{one possible child}} \quad & 
  \game{m}{\frac{6n+7-m}{2}}{\frac{6n+7-m}{2}}
\end{align*}

\bigskip
\underline{If $m \leq 2n + 2$ and $m$ is even:} \\
Consider the following move sequence:
\begin{align*}
  & \game{m}{6n+7-m}{0} \\
  \xrightarrow{\text{one possible child}} \quad & 
  \game{m}{6n+7-2m}{m} \\
  \cong \quad & 
  \game{0}{6n+7-3m}{0} \\
\end{align*}
Since $m$ is odd, we can write it as $2i$ for some $1\leq i\leq n+1$.
\begin{align*}
  & \game{0}{6n+7-3m}{0} \\
  \cong \quad & \game{0}{6n+7-6i}{0} \\
  \cong \quad & \game{0}{6(n-i)+7}{0} \\
\end{align*}
By our induction hypothesis, $\game{0}{6(n-i)+6}{0}$ has nimber $0$.

\bigskip
\underline{If $m \geq 2n + 3$:} \\
The move sequence below works:
\begin{align*}
  & \game{m}{6n+7-m}{0} \\
  \xrightarrow{\text{one possible child}} \quad & 
  \game{m}{m}{6n+7-2m}
\end{align*}

\bigskip
Which provides us with everything we need to fill in the proof
skeleton.

\paragraph{Three Point Situation, 1-Position Proof, Case 11}\mbox{}\\
Consider the initial position
\begin{equation*}
  \game{0}{6n+6}{0} \: \text{where} \: n \geq 0
\end{equation*}

\medskip
The first move brings $\game{0}{6n+7}{0}$ to a state of the form
$\game{m}{6n+7-m}{0}$, where $1 \leq m \leq 3n+3$. \\
If $m = 3n + 3$,
\begin{align*}
  & \game{3n+3}{3n+4}{0} \\
  \xrightarrow{\text{only child}} \quad & 
  \game{3n+3}{3n+3}{1}
\end{align*}

\bigskip
\underline{If $m \leq 2n + 2$ and $m$ is odd:} \\
Then, consider the following move sequence,
\begin{align*}
  & \game{m}{6n+7-m}{0} \\
  \xrightarrow{\text{one possible child}} \quad & 
  \game{m}{\frac{6n+7-m}{2}}{\frac{6n+7-m}{2}}
\end{align*}

\bigskip
\underline{If $m \leq 2n + 2$ and $m$ is even:} \\
Consider the following move sequence:
\begin{align*}
  & \game{m}{6n+7-m}{0} \\
  \xrightarrow{\text{one possible child}} \quad & 
  \game{m}{6n+7-2m}{m} \\
  \cong \quad & 
  \game{0}{6n+7-3m}{0} \\
\end{align*}
Since $m$ is odd, we can write it as $2i$ for some $1\leq i\leq n+1$.
\begin{align*}
  & \game{0}{6n+7-3m}{0} \\
  \cong \quad & \game{0}{6n+7-6i}{0} \\
  \cong \quad & \game{0}{6(n-i)+7}{0} \\
\end{align*}
By our induction hypothesis, $\game{0}{6(n-i)+6}{0}$ has nimber $0$.

\bigskip
\underline{If $m \geq 2n + 3$:} \\
The move sequence below works:
\begin{align*}
  & \game{m}{6n+7-m}{0} \\
  \xrightarrow{\text{one possible child}} \quad & 
  \game{m}{m}{6n+7-2m}
\end{align*}

\bigskip
Which provides us with everything we need to fill in the proof
skeleton.

\end{document}
