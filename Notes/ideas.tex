\documentclass{article}
\usepackage{amsmath, amssymb, mathtools, verbatim}
\usepackage{pgfplots,pgfplotstable,booktabs,array,colortbl}
\pgfplotstableset{% global config, for example in the preamble
  every head row/.style={before row=\toprule,after row=\midrule},
  every last row/.style={after row=\bottomrule},
  fixed,precision=2,
}
\usepackage[margin=0.5in]{geometry}
\setlength{\parindent}{0pt}
\renewcommand{\t}[1]{\text{#1}}
\newcommand{\N}{\mathbb{N}}
\newcommand{\sumx}[2]{\sum\limits_{#1}^{#2}}
\newcommand{\ord}{\text{ord}}
\newcommand{\frp}{\text{FRP}}
\newcommand{\R}{\Rightarrow\,}
\newcommand{\B}[1]{\textbf{#1}}
\newcommand{\I}{\bigcap}
\newcommand{\U}{\bigcup}
\newcommand{\V}{\vert}
\newcommand{\F}[1]{\lfloor#1\rfloor}
\renewcommand{\t}[1]{\text{#1}}
\newcommand{\lra}[1]{\langle#1\rangle}
\newcommand{\len}[1]{\vert#1\vert}
\newcommand{\num}{\text{num}}
\title{Combinatorial Game Theory Ideas}
\author{Kartik Sabharwal}
\begin{document}
\maketitle
\section{Five After Zero}
\subsection{Motivation}
Let $N(i)$ denote the nimber of the game with $i$ pins, where $i > 0$. \\
It so happens that for any $i$ such that $N(i) = 0$, $N(i + 1) = 5$. \\
\subsection{Patterns}
For the purposes of this section, we can turn this observation into a claim. \\
What would we need to show to prove the claim?
\begin{enumerate}
  \item we can derive a state s from $i$ such that $N(s) = 0$
  \item we can derive a state s from $i$ such that $N(s) = 1$
  \item we can derive a state s from $i$ such that $N(s) = 2$
  \item we can derive a state s from $i$ such that $N(s) = 3$
  \item we can derive a state s from $i$ such that $N(s) = 4$
  \item we can derive no state s from $i$ such that $N(s) = 5$
\end{enumerate}
So consider a number $i$ such that $N(i) = 0$. We want to show that
$i+1$ satisfies properties 1 through 6. \\
Is there a pattern in how situations where $N(n) = 5$ achieve states
that satisfy properties 1 through 5 above?
\begin{center}
\begin{tabular}{c|c c c c c} 
    & 0 & 1 & 2 & 3 & 4 \\ 
 \hline
 5  & \verb|4|  & \verb|2.2| & \verb|2| & \verb|3| & \verb|3.1| \\ 
 9  & \verb|8|  & \verb|6| & \verb|1.7,2+4| & \verb|7| & \verb|4.4,1+5| \\ 
 13 & \verb|12| & \verb|4+6| & \verb|2+8| & \verb|6.6| & \verb|10,4.8,1+9| \\ 
 17 & \verb|16| & \verb|6+8| & \verb|2+12,8.8| & \verb|6.10| 
    & \verb|1+13,3+11,4+10| \\ 
 23 & \verb|22| & \verb|?| & \verb|9+11| & \verb|21| & \verb|20| \\ 
\end{tabular}
\end{center}
\section{Record-breaking Nimbers?}
Also notice that 
$\{i : N(i) = \max\limits_{j \in [i]}(N(j))\} = \{1,2,3,5,11,19\}$
which may mean that if a given number of pins corresponds to a nimber
higher than the ones before it, the number is likely to be prime. \\
\section{Nimber-Pin Graph}
\pgfplotstabletypeset[
  columns/freq/.style={column name=Pins},
  columns/conc/.style={column name=Variant},
  columns/lino/.style={column name=Original},
]{nimberTable.txt}

\begin{figure}[h!]
\centering
\begin{tikzpicture}
\begin{axis}[
    yscale=0.5,
    grid=both,
    width=\linewidth,
    xtick=data,
    xlabel={Pins},
    ylabel=Nimber,
    legend pos=north west,
    legend entries={Variant,Original},
    ]
  \addplot table [x=pins,y=vari] {nimberTable.txt};
  \addplot table [x=pins,y=orig] {nimberTable.txt};
\end{axis}
\end{tikzpicture}
\end{figure}
\section{Expanded Nimber Graph}
\begin{figure}[h!]
\centering
\begin{tikzpicture}
\begin{axis}[
    yscale=0.5,
    grid=both,
    width=\linewidth,
    % xtick=data,
    xlabel={Pins},
    ylabel=Nimber,
    legend pos=north west,
    legend entries={Variant},
    ]
\addplot[only marks] table [x=pins,y=vari] {expandedTable.txt};
\end{axis}
\end{tikzpicture}
\end{figure}
\end{document}
